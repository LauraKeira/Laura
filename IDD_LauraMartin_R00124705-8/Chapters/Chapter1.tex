\chapter{Introduction}
\label{chap:intro}
\lhead{\emph{Introduction}}


The topic of this thesis is understanding how augmented reality and artificial intelligence can impact the education of an individual with Autism through conducting surveys, interviews and a proposed prototype design for testing. This thesis will describe in detail the current availability of computer-based instruction applications for individuals with Autism, through researching existing materials available online. There will be a comparison table of how it affected the individuals learning, if the outcome had a negative or positive impact or if the results were inconclusive. Autism Spectrum Disorders are a complex group of neurodevelopmental disabilities; in the 2016 Irish census, it has been recorded that 66,611 people or 1.4 per cent of the population, suffered from an intellectual disability, 8,902 higher than in 2011, showing a 15.4 per cent increase. The previous census information showed 10 to 14-year-old males, with 5,233 affected in this age group - more than double females (2,284)~\cite{Reference1}, the relevance to this information is that there is a rise in individuals being diagnosed with Autism and predominately in males. Studies demonstrate the difference in the estimated rates of Autism between the genders, males are more likely to be diagnosed with Autism than females. There is existing evidence to suggest that ASD prevalence rates, and needs for services, are higher in disadvantaged areas~\cite{Reference2}. This thesis will discuss a proposed prototype design based on the feedback from the surveys conducted; it is developed with trained professionals and based on school books to ensure it follows the current educational system. The study will include a table of the hypothesis for comparison and how the participants reacted to the short demos. This niche area focuses on a minority of the Irish population, it focuses again on the school-goers of the minority. The scope for this study is and research is extensive due to the ever-changing diagnosis of ASD, their change in needs and behaviours and the influence that technology can have on individuals. The studies are predominately recent as technology became more accessible and affordable globally, the rates of individuals in Ireland with ASD as seen in the 2016 census. The United States have seen an increase in Autism since 2001; the awareness of Autism is raised due to the rise in diagnosis~\cite{Reference3}. The increase in further research in Autism is because of the rise in diagnosis, meaning an increase of individuals with ASD within society with their ever-changing needs that modern society will have to adapt to and develop learning strategies for both students and teachers for the best outcome in the education of individuals with Autism. The current situation and resources based on AR, VR and AI are available to children with ASD. However, traditional teaching methods, such as books, are still the standard form of strategies used in a schooling environment. This thesis hypothesises that children with ASD respond better to computer-based instruction applications due to their social and motor limitations, asking why there is not a higher prevalence of computer-based instruction applications being used in the education of students with ASD. The increase in diagnosis and the increase of those being school-goers are possibly having their educational needs neglected due to the system for students who are developing typically, or typical is being used instead of an individual, child-centred approach being used in the classroom. The resources available in the class, the ratio of students to the teacher must be assessed and considered for future studies. The limitation of resources impacts the results. Children with Autism are considered to be visual learners. Hence, many scientifically proven audio-visual teaching approaches like Picture Exchange Communication Systems (PECS) and visual routines are incorporated into the education of Children and ASD~\cite{Reference4} Children with ASD typically respond better to computer-based instructions, RNN and CNN. Teaching children with special needs is a personalised activity that includes creating planned Individualised Education Plans (IEPs). These IEPs cover all the aspects of day-to-day life activities and tasks, from education to social interaction to personal hygiene. Along with planning and preparing IEPs, teaching includes creating lesson plans and lesson content and making logs for progress tracking. Implementing assisting aids to automate the activities mentioned above using Information and Communication Technology (ICT) will make processes more manageable and consistent. Mobile-based Augmented Reality (AR) is a multimedia-based technology that can cause computer objects to interact and engage with the Child using any handheld smart device. Augmented Reality (AR) enhances or augments reality by overlaying it with useful computer-generated graphics or virtual objects. AR applications can be of two types: marker-based or markerless AR. Markers are the optical inputs that an AR application recognises and tracks in a video stream.
AR can serve as an effective technology in developing a teaching/learning tool in real-time and increase motor skills, cognitive skills and overall increase brain development. AR can be used to track hand movement, eye movement, and behaviours~\cite{Reference5}.
% needs to be approx 1000 words
% Putting in comments within the TeX file can help make notes for yourself and dump text that you intend to edit later

\section{Motivation}
The education of Children with Autism is essential to me. I have a family member on the spectrum. I saw how his delayed development impacted his learning from a young age. Traditional education practices are not personalised towards children, especially those with learning disabilities. I have developed multiple projects for Children with special needs. As a spokesperson for people with disabilities, I decided to create a project that will positively impact children with ASD and their education.

\section{Ethical issues in the research}

\section{Research Objectives}
My research objectives are;
\begin{enumerate}
    \item To explore and research the impact AR and AI have on the Education of Autistic Children through interviews, surveys and proposed prototype design testing.
    \item Carry out primary research on the impact of computer-based instructions on Children with Autism.
\end{enumerate}

\section{Contribution}
The contribution of this research paper introduces the novelty of introducing AR and AI computer-assisted learning for Children specifically with ASD. This paper will research the use of AR and AI with no addition of physical books or QR codes. This research differs from existing information as educational professionals will be interviewed and other professionals in the field, such as SNAs and Special Needs Teachers. Surveys will be conducted with a broader range of design professionals, e-Learning designers and developers on the design and how it can be implemented into a class of Children with Autism. 

%Specify the problem. List your project goals and research questions here. Enumerate the main contributions. How have you advanced state of the art, i.e. what have you done that is new?

\section{Structure of This Document}
% notice how I cross-referenced the chapters through using the \label tag --> LaTeX is VERY similar to HTML and other markup languages, so you should see nothing new here!
%Introduce the structure of your report. This section is quite formulaic. Briefly describe this document's design, enumerating what each chapter and section stands for. For instance, in this work in Chapter \ref{chap:litreview} the guidance in structuring the literature review is given. Chapter \ref{chap:design} describes the main requirements for the problem definition and so on ...

\section{Research Methodology}

Respondents for the surveys are selected to ensure that the quality of responses is to a high standard; they are chosen based on their expertise, either in disability diagnostics, education for mainstream children and education for individuals with Autism, parents for Children with disabilities. However, the target audience for the AR testing is smaller. The feedback from surveys is on quality responses rather than on the number of responses. Further information on the data collection method is provided in the Research Methodology chapter.

\section{Prototype}

Following research on augmented reality and artificial intelligence applications and how they impacts an Autistic child's education. A prototype will be designed and based roughly on previous learning applications, the prototype was designed to simulate how AR would be presented in a classroom and how it can be used to educate children with Autism through a child-centred strategy, the Child would go at their pace, and it would be individualised lessons. Excessive amounts of research has been completed on the workings of augmented reality, including the complex technology involved and possible future developments. The research methodology chapter identifies how the data is collected for the research and why this particular method was chosen. The prototype and design testing chapter explains the path taken to develop the prototype from concept to the final version. The results and analysis chapter examines the responses from the four surveys and seven interviews. These findings are then analysed in the Results and Analysis chapter. Further discussion on the methodology chosen is then given in the Discussion chapter. To conclude at the end of the thesis, a conclusion provides some final thoughts on the research and further work and development. 


