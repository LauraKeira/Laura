\chapter{Research Methodology}
\label{chap:Research Methodology}
\lhead{\emph{Research Methodology}}

\section{Research Methodology}
This research methodology will use both qualitative and quantitative methods 
for analysing. There will be two surveys carried out online to get varied feedback. The first survey will be for educational professionals who educate Children with Autism. The second survey conducted online will be for individuals who have family members, themselves, who have close contact or Child who is Autistic. It is vital to get feedback from both educational professionals and non-educational professionals concerning their understanding of how applications with artificial intelligence and augmented reality can educate children with Autism and other learning disabilities. The first survey, "Professional Educator Survey", will consist of 30 questions for identifying careers and ranks among the participants, their previous knowledge of AR and AI, how technology is used in the class and questions about students ability and possible limitations when it comes to resources available in the school to use technology in the classroom. The second survey, "Autism and Assistive Technology," will have participants who are not educational professionals; they have a personal relationship with a Child with Autism. It is essential to get feedback from individuals with no teaching background or experience; this feedback can be used to design the prototype so the application can be used both for schoolwork and homework.


There will be interviews with three individuals from the educational professionals, two special need teachers and one special needs assistant on the proposed prototype design created in Adobe Illustrator. In conjunction, there will be three interviews with Parents who have a primary school child who is Autistic on the proposed prototype and to gain feedback on the design to make it user friendly for both educational professionals and non-educational professionals. There will be multiple interviews and surveys to get a varying amount of information and feedback on the prototypes and how they can be improved and then implemented into a person with Autism and their education. A person with Autism can have different patterns and behaviours to another person with Autism; it is essential to get feedback from a sample that ranges in age, demographic and employment within the professional educator sector. In order to compare and gain more insight into the topic of AR and AI, existing information from journals, surveys and online content will also be researched and included in this paper. Concerning the information being sourced online, there will have to be things to consider, such as; date range, credibility, sample size being surveyed, and tested. The date range will be significant as more recent information is needed for accurate results due to ongoing upgrades and new products being introduced in technology and education of Children with ASD. Discourse analysis will have to be considered, as looking at communication and meaning (languages, images, non-verbal interactions) will be taken into account for social context.

A custom-built AR filter using Spark and a 3D modelling software will be used in five participants, both males and females with Autism, these participants will be of school-going age, and their abilities will be different to one another. This testing will include two non-verbal Children with Autism. They are currently being introduced to other communication learning applications in their school and home to help with their current communication difficulties. Finally, for prototype testing, a short demo video of the proposed design will be created in After Effects and shown to both participants who were involved in the interviews, the parents from the surveys who have a Child with Autism and the Autistic individuals to get feedback and to be able to analyse results.




The following will be taken into consideration during testing.
\begin{itemize}
    \item Control - The same number of activities and time will be given to each participant. 
    \item Sample Size - The sample size will include both males and females; the participants will vary in age and demographics.
    \item Replicates - The same experiment will be given to each sample of participants depending on their severity of the disability.
\end{itemize}

Participants will be found through an ASD unit; all Autistic participants will be of school-going age as this is a formal operational stage for typical Children. School going age in typical children can use logic to solve problems, view the world around them, and plan. [6]. Groups will be small due to a lack of access to a large group of Children with Autism and this will give precise and accurate information for the data analysis. When all surveys, interviews and testing has been conducted, there will be a section for analysis on the results and a conclusion of the research findings. 

While conducting user interviews and user testing to develop standard operating procedures (SOPs) that will align with the proposed design.  Interviews will be based on questions to determine what key  design  factors are beneficial to the proposed application while  using  an  SOP  and  how  they  perceive their performance  while interacting with an SOP.