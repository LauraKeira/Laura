Autism spectrum disorders are a group of complex neurodevelopmental conditions that affected individuals social interactions, communication, and repetitive and stereotyped behaviours and patterns. Computer-based instructions and technology in the classroom may affect the development of social skills among students, especially those on the autism spectrum and potentially help with their education and learning phases. There are various types of technology used in the classroom including gamification, assistive technologies, interfaces, portable devices, robotic devices, software, and videos. At the same time, focusing on instructional technology and assistive technology and how they can positively impact children with Autism and their education. Technology can allow computer-based instructions to be repeated; this can benefit an Autism Child who has repetitive behaviours and patterns; meaning the Children are not overwhelmed as they can engage and interact at their own pace and level of learning. Research has been carried out in diagnosing and early intervention of a child with Autism; there is a lack of research in the formal education of Autistic Children and technology. As technology use, design, devices and target markets, meaning gaps in technology for Autistic children. The untapped potential in everyday potential technological devices as educational aids and tools could be huge for children with Autism.

Line of how i am different